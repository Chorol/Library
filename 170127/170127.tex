\documentclass[paper=a4, fontsize=11pt,twoside]{article}

% --------------------------------------------------------------------
% General Page Layout
% --------------------------------------------------------------------
\usepackage[a4paper]{geometry}
\usepackage[parfill]{parskip}
\setlength{\oddsidemargin}{5mm}		% Remove 'twosided' indentation
\setlength{\evensidemargin}{5mm}
%%
% --------------------------------------------------------------------
% Encoding and Language Settings
% --------------------------------------------------------------------
\usepackage[T1]{fontenc}
\usepackage[utf8]{inputenc}			% encoding may need to be changed depending on the system
\usepackage[swedish]{babel}
\usepackage{lipsum}					% Lorem Ipsum

% --------------------------------------------------------------------
% Utilities (colors, links, pictures, ect...)
% --------------------------------------------------------------------
\usepackage{xcolor}
\usepackage{hyperref}
\usepackage{graphicx}
\usepackage{amssymb}
\usepackage{epstopdf}
\usepackage[round]{natbib}
\usepackage{float}
\DeclareGraphicsRule{.tif}{png}{.png}{`convert #1 `dirname #1`/`basename #1 .tif`.png}

\title{Möte}
\author{Grupp 2}
\date{27-01-17}

\begin{document}
\maketitle	

\section{Frånvarande}
\begin{itemize}
  \item Erik Rosenström
  \item Maurits Johansson
  \item Simon Farre
\end{itemize}

\section{Agenda}

\begin{itemize}
  \item Discord
  \item GitHub
  \item E-PUSS
  \item Laborationer
  \item Granskning
  \item Ingenjörsinformation
\end{itemize}

\section{Protokoll}

\subsection{Discord}
\begin{itemize}
  \item Alla i gruppen ska nu finnas med i kommunikationsapplikationen
Discord.
  \item Under mötet fanns en önskan om att ha en godtycklig
kommunikationssätt inom gruppen, inicialt genom att tillämpa all form av digitaliserad
  kommunkation genom en enda kommunikations- länk, istället för att tillämpa
  mailkommunikation i kombination med applikationsbaserad kommunikation.
  \item Inom mötet fanns en enighet i att använda endast Discord för all
  digitaliserad kommunikation. 
  \item Med avseende på det ovannämnda, är mailkommunikation från och med
dagens datum inte längre den medtod som ska tillämpas inom projektet för kommunikatinosendamål. All kommunikation sker via Discord.
  \item Den tidigare satta ansvaret (att kolla mailet minst en gång per
dygn) har nu istället blivit förflyttad till att gälla för Discord. Alla inom projektet ska kolla minst en gång per dygn sitt Discord – helst oftare.
\end{itemize}



\subsection{GitHub}
\begin{itemize}
  \item Git-dokummentbiblioteket är nu klar för invigning. Vi ska inom projektet börja använda Git redan i dag för 
  att säkerställa att all synkronisering samt sammanparning av dokumment fungerar som förväntat för att 
  eliminera eventuell problematik när vi väl går över i fas 2 och ska börja tillämpa Git i vart dagliga arbete.
  \item Inicialt kan versionshanteringssystemet uppelvas som något invecklat och svårt, framför allt för användare
  som inte har någon tidigare erfarenhet av systemet. Dock bör det påpekas att efter lite praktisk övning så
  är man snabbt igång och får en större trygghet för systemet. Därför är det viktigt att vi redan nu använda
  GitHub för att kunna lära upp oss innan vi går över till ett mer kritisk skede där det inte finns utrymme 
  för felaktig användning
\end{itemize}

\subsection{E-PUSS}
\begin{itemize}
  \item Veckorapportering för föregående vecka signeras av PL på måndagar. Veckorapportering som kommer in efter
  signeringsdagen kommer inte bli signerade och kommer inte kunna tillgodoräknas i projektet, d.v.s 
  det betraktas som frånvaro. Därför är det viktigt att veckorapporteringen görs i god tid så att den hinner
  bli signerad av PL på måndag.
\end{itemize}

\subsection{Laborationer}
\begin{itemize}
  \item Vi har en intensiv schema med många laborationer, men vi får ta förgivet att projektet kommer fortskrida av
  sig själv. Många hade laboration i dag, men de som var fria behövdes inom projektet men fanns inte närvarande.
  Vi har därför diskutterat under mötet hur mycket tid vi behöver avlägga för laborationer i OMD
  och kommit fram till att det kommer gå åt två (2) timmar per vecka för detta endamål. Skulle man behöva
  längre tid får man avsätta det på tider utanför projektets arbetstid. Projektets arbetstid deffineras som 
  den tid mellan kl. 8 fram till kl. 16 – med undantag för lunch mellan kl. 12-13.
\end{itemize}

\subsection{Granskning}
\begin{itemize}
  \item Vi befinner oss i situation där dokumenten för fas 1, d.v.s. SDP, SRS och SVVS, behöver bli klara redan i dag,
  för att kunna genomföra en inofficiell granskning nästa vecka och hinna genomföra eventuella ändring samt 
  genomföra inofficiell omgranska innan vi genomför den officiella granskningen på måndag vecka 6.
  \item För att vi ska kunna bli klara med dokummenten i dag, behöver vi omkoordinera oss. Samtliga inom UG hjälper 
  till med SRS:n och detta görs nu under dagen, och eventuellt under helgen.
\end{itemize}

\subsection{Ingenjörsinformation}
\begin{itemize}
  \item Det har kommunicerats ut information om ingenjörsinformationen som kommer hållas av Campus beträffande
  elever som önskar söka till senare delen av civilingenjörsutbildningen. Förfrågningen var hur många
  som var interesserade att delta på denna information, men få hade svarat.
\end{itemize}


\end{document}